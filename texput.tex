\title{Sampling of Signals}
\author{Vicente González Ruiz}
\maketitle
\tableofcontents

\section{Nyquist–Shannon Sampling Theorem~\cite{oppenheim2014discrete,lathi1998modern}}
\begin{itemize}
\item Defines the requisites of the sampling process to be sucessful.
\item Let $s(t)$ a band-width limited signal, with maximum frequency
  component $f_{\text{m}}$ Hz. The Sampling Theorem says that, in order
  to sample $s(t)$, $2f_{\text{m}}$ samples per second must taken.
  $2f_{\text{m}}$ is known as the Nyquist Frequency.
\end{itemize}

\section*{Proof}
\begin{itemize}
\item According to the statement, we have that $S(\omega)=0$ when
  $|\omega|>\omega_{\text{m}}$ (it is limited in band), where
  $\omega_{\text{m}}=2\pi f_{\text{m}}$ is the maximum angular frequency
  of $s(t)$.
  \svg{ft}{800}

\item The sampling process of $s(t)$ can be modeled mathematically by
  the multiplication of $s(t)$ with a train of unit impulses
  $\delta_{T_{\text{s}}}(t)$.
  \svg{TFtren_de_impulsos}{800}

\item Let

  \begin{equation}
    s_{\text s}(t)=s(t)\delta_{T_{\text s}}(t)
  \end{equation}

  the resulting sampled signal, where (see definition of $\delta_T$
  in \href{http://localhost:8888/notebooks/Multimedia_Systems/Digitalization/Harmonic_Analysis/harmonic_analysis.ipynb#Fourier-Transform-of-train-of-equidistant-unit-impulses}{Fourier
  Transform of train of equidistant unit impulses ($\delta_T(t)$)})

  \begin{equation}
    \delta_{T_{\text s}}(t) = \sum_{n=-\infty}^{\infty}\delta(t-nT_{\text s})
  \end{equation}

  and

  \begin{equation}
    T_{\text s}=\frac{2\pi}{\omega_{\text s}}
  \end{equation}

  is the sampling period, being

  \begin{equation}
    \omega_{\text s} = 2\pi f_{\text s}
  \end{equation}

    the angular sampling frequency, expresed in radians/second.

\item As we also known from \href{http://localhost:8888/notebooks/Multimedia_Systems/Digitalization/Harmonic_Analysis/harmonic_analysis.ipynb#Fourier-Transform-of-train-of-equidistant-unit-impulses}{Fourier Transform of train of equidistant unit impulses}, that:
  \begin{equation}
    {\mathcal F}[\delta_{T_{\text s}}(t)]=\Delta_{T_{\text s}}(\omega)=\omega_{\text s}\delta_{\omega_{\text s}}(\omega).
  \end{equation}

  Considering that the multiplication of two functions in the time domain is equivalent to the convolution of their spectrums, it holds that

  \begin{equation}
    S_{\text s}(\omega) = \frac{1}{2\pi}\big(S(\omega)*\omega_{\text s}\delta_{\omega_{\text s}}(\omega)\big).
  \end{equation}

  After substituting $\omega_{\text s}=\frac{2\pi}{T_{\text s}}$, we get that

  \begin{equation}
    S_{\text s}(\omega) = \frac{1}{T_{\text s}}\big(S(\omega)*\delta_{\omega_{\text s}}(\omega)\big)
  \end{equation}

  which by definition of $\delta_{\omega_{\text s}}(\omega)$, is

  \begin{equation}
    = \frac{1}{T_{\text s}}\big(S(\omega)*\sum_{n=-\infty}^{\infty}\delta(\omega-n\omega_{\text s})\big)
  \end{equation}
  \begin{equation}
    = \frac{1}{T_{\text s}}\sum_{n=-\infty}^{\infty}S(\omega)*\delta(\omega-n\omega_{\text s})
  \end{equation}
  \begin{equation}
    = \frac{1}{T_{\text s}}\sum_{n=-\infty}^{\infty}S(\omega-n\omega_{\text s}).
  \end{equation}

  Therefore, the spectrum of $s_{\text s}(t)$ is equal to the replica  of the spectrum of $s(t)$ every $\omega_{\text s}$ radians/seconds.

  \svg{fst}{800}

\item As it can be seen, it is possible to reconstruct $S(\omega)$ from $S_{\text s}(\omega)$ if the frequencies higher than $\omega_{\text m}$ are discarded (filtered out), multipliying the spectrums

  \begin{equation}
    S(\omega) = S_{\text m}(\omega)T_{\text s}G_{2\omega_{\text m}'}(\omega)
  \end{equation}

  where the rectangular signal is

  \begin{equation}
    T_{\text s}G_{2\omega_{\text m}'}(\omega) =
    \left\{
    \begin{array}{ll}
      T_{\text s} & \text{if $|\omega| < \omega_{\text m}'$} \\
      0 & \text{otherwise}
    \end{array}
    \right.
  \end{equation}

  \svg{pulso}{250}

  provided that $$\omega_{\text m}\leq \omega_{\text m}'.$$

\item $T_{\text s}G_{2\omega_{\text m}'}(\omega)$ is the Fourier transform is the Sinc signal

  \begin{equation}
    2T_{\text s}\omega_{\text m}'{\text Sinc}(\omega_{\text m}'\omega)
  \end{equation}

  which is, by definition, the response of a low-pass filter, with cut-off frequency $\omega_{\text m}'$ and gain $T_{\text s}$, to the unit impulse, also known as the transfer function of the filter. Therefore, it also holds that
  \begin{equation}
    s(t) = s_{\text s}(t)*2T_{\text s}\omega_{\text m}'{\text Sinc}(\omega_{\text m}'\omega).
    \tag{convolution\_sampling}
  \end{equation}

\item The effect produced with $\omega_{\text m}' < \omega_{\text m}$ is
  known as \emph{aliasing}.  \bibliography{text-compression}

  \svg{fst2}{400}

  In this situation, it is impossible to reconstruct $s(t)$ from $s_{\text s}(t)$ because $S(\omega)$ can not be extracted from $S_s(\omega)$.

\end{itemize}

\section{Time-limited (finite) sampled signals have frequency-unlimited (infinite) spectrums}
\begin{itemize}
\item Lets suppose $f_{\text m}$ is the maximum frequency of the signal $s(t)$.

\item A finite signal $s'(t)$ can be obtained from $s(t)$ by multiplying:

  \begin{equation}
    s'(t) = s(t)w_\tau(t)
    \tag{time-domain\_windowing}
  \end{equation}

  where $w_\tau(t)$ (that in its simplest shape is a rectangular signal) is commonly called, *Window signal* or *Window function*.

\item Considering the convolution theorem, Eq. (time-domain\_windowing) equals to:

  \begin{equation}
    S'(\omega) = S(\omega)*W_\tau(\omega)
    \tag{frequency-domain\_windowing}
  \end{equation}

  where $W_\tau(\omega)$ is (again, in its simplest shape) a Sinc
  signal, that by definition if bandwidth unlimited (except if the
  duration of the window is infinite).
\end{itemize}


\section{Minimizing the spectral leakage~\cite{lathi1998modern}}

\begin{itemize}
\item The shape of $W_\tau(\omega)$ affects to the shape of the
  windowed signal $S'(\omega)$: the higher the resemblance between
  $W_\tau(\omega)$ and a Delta signal, the lower the deformation of
  $S'(\omega)$ compared to $S(\omega)$.

\item Some famous Window functions are the follow:
  \begin{enumerate}
  \item Rectangular Window:
    \begin{equation}
      w_\tau(t) =
      \left\{
      \begin{array}{ll}
        1 & \text{if $|t|<\frac{\tau}{2}$}\\
        0 & \text{otherwise}
      \end{array}
      \right.
    \end{equation}

  \item Hanning Window:
    \begin{equation}
      w_\tau(t) =
      \left\{
      \begin{array}{ll}
        0.5-0.5\cos(2\pi t) & \text{if $|t|<\frac{\tau}{2}$}\\
        0 & \text{otherwise}
      \end{array}
      \right.
    \end{equation}

  \item Bartlett (triangular) Window
    \begin{equation}
      w_\tau(t) =
      \left\{
      \begin{array}{ll}
        2t & \text{if $-\frac{\tau}{2}<t\le 0$}\\
        -2t & \text{if $0<t\le \frac{\tau}{2}$}\\
        0 & \text{otherwise}
      \end{array}
      \right.
    \end{equation}
    
  \end{enumerate}
  
\end{itemize}

\subsection*{Comparison}
\begin{itemize}
\item Time:
  \svg{ventanas}{800}
\item Frequency:
  \svg{espectro_ventanas}{800}
\end{itemize}

\bibliography{signal-processing}
